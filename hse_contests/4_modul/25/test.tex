% Проверьте, что вы сохранили файл в кодировке utf8
% encoding: utf-8
%

\documentclass[12pt]{article} % 12 – размер шрифта

% Оптимизированное подключение необходимых пакетов
\usepackage[utf8]{inputenc} % Кодировка файла
\usepackage[T2A]{fontenc} % Поддержка кириллицы
\usepackage[russian]{babel} % Русские переносы и оформление
\usepackage{graphicx} % Вставка изображений
\usepackage{amsmath, amssymb, amsfonts} % Математические символы и шрифты
\usepackage{dsfont} % Дополнительные символы
\usepackage{hologo} % Логотип LaTeX
\usepackage[russian]{hyperref} % Гиперссылки
\usepackage{listings} % Для вставки кода
\usepackage{xcolor} % Работа с цветами
\usepackage{float} % Плавающие объекты
\usepackage{wrapfig} % Обтекание фигур
\usepackage{hyperref}
\usepackage{bbold} % Дополнительные символы
\usepackage{titlesec} % Для настройки заголовков секций
\usepackage[margin=2cm]{geometry} % Настройка полей страницы
\documentclass{standalone}
\usepackage{tikz}
\usepackage{forest}




% Определение новых цветов
\definecolor{lightblue}{RGB}{173, 216, 230} % Светло-синий
\definecolor{codegreen}{rgb}{0,0.6,0}
\definecolor{codegray}{rgb}{0.5,0.5,0.5}
\definecolor{codeblue}{RGB}{59,0,255}
\definecolor{backcolour}{RGB}{245,245,245}

\usepackage{tikz}





\usepackage{tcolorbox}
% В преамбуле:
\newtcolorbox{myquote}{colback=white!10!white, colframe=gray!80!black, size=title, leftrule=3mm}




\title{Домашнее задание по Дискре  }
\author{Даниил Калашников}
\date{April 2025}

\begin{document}


\maketitle
\section*{Задание 1}
\begin{enumerate}
    \item Первое , что давайте сделаем , давайте обозначим всех  двоечников $U$ . За $X$ мы обозначим двоечников , которые ходят на кружки ( не лентяев) , так как их конечное кол-во , если мы их выкненм , то у нас все равно останется бесконечное кол-во двоечников , только теперь остануться лишь те , которые ходят на кружки . 
    

    \item Давайте возьмем какую-то компанию $G$ , и пересечем ее с каким-то составом кружка $K_1$ , по условию у нас получиться , что их объединение будет не пустым то есть $|G \cap K_1| \geq |2|$ , а также оно будет счетным (пересечение бесконечноего и конечного). Таких групп у нас по условию бесконечное кол-во , а значит у нас будет образовываться бесконечное кол-во пар( троек и тд) людей . 

    \item Пусть найдется какой-то двоечник ,  который записан в конечное кол-во кружков , тогда у него было конечное количество пар (троек и тд ) , в которое он входил, но так как пар( троек и тд ) бесконечно много получатеся противоречие, значит каждый из двоечников был записан в бесконечное множество кружков. 
    
\end{enumerate}

\section*{Задание 2}
\subsection*{а)}
\begin{myquote}

     \textbf{Idea:} Для того , чтобы доказтаь , данное утвреждение , необходимо показать , что одно множество , можно вложить в другое и наоборот , также необходимо предъявить биекцию между $O_p \rightarrow O_2$ , и также с другой , тем самым мы сможем сделать так , что у нас они будут равномощны .
\end{myquote}

\begin{enumerate}
    \item Мы можем числа $\{1 \dots p\}$ , $\{1 \dots q\}$ представить в виде чисел в $p$-той и $q$-той степени, мы знаем , что можно любую систему счисления можно привести в $2$-ичную (используя дерево кодирования  Хаффмана ) , таким образом у нас есть инъекция между $O_p \twoheadrightarrow O_2$ и $O_q \twoheadrightarrow O_2$  
    \item Также используя дерево декодирования Хаффмана , можно числа из $2$-чной перевести в другую степень , поэтому есть инъекция в другую строну $O_2 \twoheadrightarrow O_p$ и $O_2 \twoheadrightarrow O_q$ 
    \item Получается у нас справедливо вот это $O_2 \subseteq O_p , O_2 \subseteq O_q , O_p \subseteq O_2 , O_q \subseteq O_2$ , а из этого по теорема Кантора-Берштейна следует то , что $|O_p| = |O_q|$
\end{enumerate}


\subsection*{b)}

\begin{myquote}
    \textbf{Idea:} Хмм нам нужно беконечную последоваткельность из чисял закодировать элементами из O 2 ,  а давайте возьмем и закодируем цифры у нас всего их 10 получается нам необходимо 2 байта , на то , чтобы это сделать , то есть 
\end{myquote}




\begin{enumerate}
    \item Давайте возьмем и каждое из чисел закодируем деревом Хаффмана , получиться , что максимальная длина у закодированного символа будет 4 . 
    \centering \begin{forest}
for tree={
    draw,
    rounded corners,
    node options={align=center},
    edge={->, >=latex},
    parent anchor=south,
    child anchor=north,
}
[Root
    [0 (011)] 
    [1(110)]
    [2(101)
        [3(0100)]
        [4(1111)]
    ]
    [5 (001)
        [6(1110)]
        [7(1001)]
    ]
    [8 (1000)
        [9(000)]
        [| (0101)]
    ]
]
\end{forest}

    \item После чего давайте научимся составлять большие числа из закодированных в двоичной системе . Будем делать так : 
    \begin{enumerate}
        \item Длина числа будет задаваться набором $1$ , и их кол-вом 

        \item Для того , чтобы обозначать что начинается число , мы в конец записи их единиц будем добавлять 0 , а также в конце ввода давайте ставить особый символ закончания , на который мы сразу закодируем 
        
        \item Пример:
        $$ \{10 , 5  \} \rightarrow \underbrace{11}_{\text{Длина}} \underbrace{0}_{\text{Разделитель}} \underbrace{110 011}_{\text{число}}  \underbrace{0101}_{\text{конец числа}} \underbrace{1}_{\text{Длина}} \underbrace{0}_{\text{Разделитель}} \underbrace{001}_{\text{число}} \underbrace{0101}_{\text{конец числа}} $$
        \item Таким образом , мы постриили биекцию между $O_2 \hookrightarrow O_N$
    \end{enumerate}
\end{enumerate}

\subsection*{с)}
\begin{myquote}
    Для того, что решить данную задачу , можно 
    \begin{enumerate}
        \item показать , что послдеовальность из R чисел нельзя закодировать 2 . получается что и $O_2$ оно не равномощно 
        \item тогда оно будет выглядеть так : изначальная длина делителя  + разделитель + кол-во знаков у делимого  + разделитль + набор цифр делителя  + конец ввода  цифр делимого + набор цифр делимого + конец числа
    \end{enumerate}
\end{myquote}
Такс , для начала определим их: 
\[
\Theta_{\mathbb{N}} = \{ (a_1, a_2, \dots) \mid a_i \in \mathbb{N} \}, \quad \Theta_{\mathbb{R}} = \{ (x_1, x_2, \dots) \mid x_i \in \mathbb{R} \}
\]
\begin{enumerate}

    \item Давайте возьмем произвольную последовательность \( (a_n)_{n \in \mathbb{N}} \in \Theta_{\mathbb{N}} \). Поскольку \( \mathbb{N} \subset \mathbb{R} \), то можно отобразить \( (a_n) \mapsto (a_n) \in \Theta_{\mathbb{R}} \). Это отображение — инъекция, получается \( \Theta_{\mathbb{N}} \twoheadrightarrow \Theta_{\mathbb{R}} \):
    
    \item Покажем, что \( \Theta_{\mathbb{R}} \twoheadrightarrow \Theta_{\mathbb{N}} \):\\
    Каждое вещественное число \( x_i \in \mathbb{R} \) можно представить в виде бесконечной двоичной дроби:
    \[
    x_i = \pm (a_0 + 0.a_1a_2a_3\dots)_2, \quad a_j \in \{0, 1\}
    \]
    (мы выбираем представление без хвоста из бесконечных единиц). Каждую такую бесконечную последовательность можно закодировать последовательностью натуральных чисел — например, разбив её на блоки фиксированной длины и переводя в числа . Таким образом, каждое \( x_i \) отображается в элемент \( \Theta_{\mathbb{N}} \), а значит и вся последовательность \( (x_i) \in \Theta_{\mathbb{R}} \) также отображается в \( \Theta_{\mathbb{N}} \). Это и будет инъекция.

    \item По теореме Кантора–Бернштейна из наличия инъкции в одну и в другую сторону , мы получаеем , что их мощности равны , и она равна континууму:
    \[
    |\Theta_{\mathbb{N}}| = |\Theta_{\mathbb{R}}| = \mathfrak{c}
    \]
\end{enumerate}

\subsection*{Задание 3}
\begin{myquote}
    \textbf{first Idea : } Давайте возьмем какую-то точку , для нее , существует бесконечно много радиусов , также каждый из радиусов задает какой-то бескоенчный набор точек , которые лежат на ридусе .(и этот набор не являтеся счетным)- так как у нас плоскость представляет их себя множество $Q \times Q$  , где Q - мн-во иррациональных;
    Раз точек не более чем счетное множество то  даже если мы расположим одну точку в круге , который образован счетным мноежством , а другую точку вне этого круга , тогда за счет того , что между точками сченого мноежства буедт некое расстояние , можно будут соеденить эти две точки 
\end{myquote}


\begin{myquote}
    \textbf{second Idea  : }
    \begin{enumerate}
        \item Давайте зафиксируем две непомеченные точки \( A \) и \( B \).
        \item Тогда все возможные центры дуг окружностей , которые будут проходить через них ,  будут лежать на серединном перпендикуляре к отрезке ($A-B$) 
        \item На данном перпендикуляре несчетное кол-во точек , так как плоскость состоит из $R \times R$  , где R - множество вещенсвтенных чисел ц
        \item А центров , для которых их дуги будут пересекать помеченные точки , будет счетное кол-во , в силу этого , остается все равно несчетное кол-во точек , которые будут допустимыми центарми . 
        \item получается , что можно выбрать такой центр окружности, что соответствующая дуга через \( A \) и \( B \) не будет проходить через помеченные точки.

    \end{enumerate}

\end{myquote}
\section*{Задание 4}
\subsection*{a)}
\begin{myquote}
Такс я узнал , что мощность книутиуума имеют : 
    \begin{enumerate}
        \item множесвтво вещенствекнных чисел 
        \item множество всех подмножеств счетного мнгожества , то есть допустим $2^N$
    \end{enumerate}
Тогда в нашем случае \( \mathbb{R} \) имеет мощность континуума: \( |\mathbb{R}| = \mathfrak{c} \). Тогда множество функций из \( f : \mathbb{R} \to \mathbb{R} \) будет из себя предствялть множество всех отображений из  \( \mathbb{R} \) в \( \mathbb{R} \). А его мощность равна:
    \[
        |\mathbb{R}^{\mathbb{R}}| = \mathfrak{c}^{\mathfrak{c}} > \mathfrak{c}
    \]
    То есть оно будет  строго больше мощности континуума.
\end{myquote}

\subsection*{b)}
\begin{myquote}
    
\begin{enumerate}
    \item Такс , у нас \( \mathbb{R} \) имеет мощность континуума: \( |\mathbb{R}| = \mathfrak{c} \). 
    \item Биекция представляет их себя подмножество всех фукнций , мощность которого будет равна $2^\mathfrak{c}$ , так как биекцию можно представлять как мнгожество перестановок элементов $\mathbb{R}$, а кол-во перестановок множества мощности $\mathfrak{c}$ будет $2^\mathfrak{c}$
    \[
        |f: \mathbb{R} \hookrightarrow \mathbb{R} , \text{где } f - \text{биекция}| = 2^\mathfrak{c}
    \]
\end{enumerate}
\end{myquote}
\subsection*{с)}
\begin{myquote}
    
Что вообще можно скзатаь о множестве непрервывных функций ? - мб что оно счетно , тогда так как  множество всех подмножеств счетного мнгожества , являтеся равномощным мощности континиуума , у нас получается , что 
\[
        |2^\text{Множество непрерывных функций }| = \mathfrak{c}
\]
\end{myquote}

\section*{Задание 5}

\textbf{Для пяторок: } Давайте их будет распологать на вещественной прямой таким образом, чтобы их прямх 


\end{document}


